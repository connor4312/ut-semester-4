\documentclass{article}
\usepackage{amsmath,amssymb,graphicx,tikz}
\usepackage{hyperref}
\usepackage{tikz}
\usepackage{amsfonts}
\usepackage{amsmath}
\usepackage{forest}
\usepackage{listings}
\usetikzlibrary{fit}

\title{CSC263 Winter 2016, Assignment 2}
\author{Connor Peet \#1001088208}
\renewcommand{\today}{~}
\hypersetup{pdfpagemode=Fullscreen,
  colorlinks=true,
  linkfileprefix={}}
\begin{document}
\maketitle

\lstset{
    numbers=left
}

\begin{enumerate}
\item [1.]
    \begin{enumerate}
    \item See Figure 1 below.
        \begin{figure}[p]
        \centering
        \begin{forest}
        for tree={
            align=center,
            edge path={
                \noexpand\path [draw, \forestoption{edge}] (!u.parent anchor) -- +(0,-15pt) -| (.child anchor)\forestoption{edge label};
            },
            if n=2{
                if ={equal(n_children("!u"),3)}{calign with current}{}}{},
        }
        [BFS
            [{p\\\scriptsize{$d[p] = 1$}}
                [{o\\\scriptsize{$d[o] = 2$}}
                    [{r\\\scriptsize{$d[r] = 5$}}
                        [{u\\\scriptsize{$d[u] = 7$}}
                            [{t\\\scriptsize{$d[t] = 11$}}]
                        ]
                        [{y\\\scriptsize{$d[y] = 8$}}]
                    ]
                    [{v\\\scriptsize{$d[v] = 6$}}
                        [{w\\\scriptsize{$d[w] = 9$}}]
                        [{x\\\scriptsize{$d[x] = 10$}}]
                    ]
                ]
                [{s\\\scriptsize{$d[s] = 3$}}]
                [{z\\\scriptsize{$d[z] = 4$}}]
            ]
            [{m\\\scriptsize{$d[m] = 12$}} [{q\\\scriptsize{$d[q] = 13$}}]]
            [{n\\\scriptsize{$d[n] = 14$}}]
        ]
        \end{forest}
        \caption{The resulting BFS tree $T$.}
        \end{figure}


    \item See Figure 2 below.
        \begin{figure}[p]
        \centering
        \begin{forest}
        for tree={
            align=center,
            edge path={
                \noexpand\path [draw, \forestoption{edge}] (!u.parent anchor) -- +(0,-20pt) -| (.child anchor)\forestoption{edge label};
            },
            if n=2{
                if ={equal(n_children("!u"),3)}{calign with current}{}}{},
        }
        [DFS
            [{m\\\scriptsize{$d[m] = 1$}\\\scriptsize{$f[m] = 20$}}
                [{q\\\scriptsize{$d[q] = 2$}\\\scriptsize{$f[q] = 5$}}
                    [{t\\\scriptsize{$d[t] = 3$}\\\scriptsize{$f[t] = 4$}}]
                ]
                [{r\\\scriptsize{$d[r] = 6$}\\\scriptsize{$f[r] = 19$}}
                    [{u\\\scriptsize{$d[u] = 7$}\\\scriptsize{$f[u] = 8$}}]
                    [{y\\\scriptsize{$d[y] = 9$}\\\scriptsize{$f[y] = 18$}}
                        [{v\\\scriptsize{$d[v] = 10$}\\\scriptsize{$f[v] = 17$}}
                            [{w\\\scriptsize{$d[w] = 11$}\\\scriptsize{$f[w] = 14$}}
                                [{z\\\scriptsize{$d[z] = 12$}\\\scriptsize{$f[z] = 13$}}]
                            ]
                            [{x\\\scriptsize{$d[x] = 15$}\\\scriptsize{$f[x] = 16$}}]
                        ]
                    ]
                ]
            ]
            [{n\\\scriptsize{$d[n] = 21$}\\\scriptsize{$f[n] = 26$}}
                [{o\\\scriptsize{$d[o] = 22$}\\\scriptsize{$f[o] = 25$}}
                    [{s\\\scriptsize{$d[s] = 23$}\\\scriptsize{$f[s] = 24$}}]
                ]
            ]
            [{p\\\scriptsize{$d[p] = 27$}\\\scriptsize{$f[p] = 28$}}]
        ]
        \end{forest}
        \caption{The resulting DFS tree $T$.}
        \end{figure}
    \item There are a couple cases:
        \begin{itemize}
        \item $d[a] < d[b]$: then $a$ or one of its children will have discovered $b$. There will either be a tree edge or a forward edge between them. In this case, it will always be that $d[a] < d[b] < f[b] < f[a]$.
        \item $d[a] > d[b] \land f[a] > f[b]$: then there will be a cross-edge. $b$ and $a$ did not discover each other.
        \item $d[a] > d[b] \land f[a] < f[b]$: then there will be a back edge between them. That is, the search from $b$ transversed a cycle which caused $b$ to discover its parent $a$.
        \end{itemize}
    \item From Lemma 22.11, we know a graph is acyclic if its DFS has no back edges. The generated DFS did not have any back edges visually or based on the definition above, so it's acyclic.
    \item $\{ t, q, u, z, w, x, v, y, r, m, s, o, n, p \}$
    \end{enumerate}
\end{enumerate}
\end{document}
