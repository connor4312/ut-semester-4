\documentclass{article}
\usepackage[american]{babel}
\usepackage[backend=bibtex, style=mla, annotation=true]{biblatex}
\bibliography{lin200-proposal}
\usepackage{hyperref}
\usepackage{amsmath}
\usepackage{listings}

\title{LIN200 Winter 2016, Research Proposal}
\author{Connor Peet \#1001088208}
\renewcommand{\today}{~}
\hypersetup{pdfpagemode=Fullscreen,
  colorlinks=false,
  linkfileprefix={}}
\begin{document}
\maketitle

In my paper, I will argue that there is no definition of a word which is universal. Specifically, I'll contrast the concept of a word in English and Mandarin, evaluating the proposed definitions that Heidi Harley puts for in her book \textit{What is a Word?} \cite{words}. These rather fragile definitions are shattered when attempting to apply them to languages syntactically different than English, such as Chinese \cite{mandarin}. Harley proposes that a word is a sequence of letters written consecutively without spaces, but Chinese is logographic and does not use spaces \cite{mandarin}. She alternately asserts that a word is ``a sequence of sounds which \textit{can} be pronounced on its own, with space on either side,'' (emphasis original), but that does not require them to be meaningful and is problematic when dealing with compound words. Finally, she proposes a longer definition which states that a word is a combination of sounds that expresses an idea, but this is ambiguous when inspecting transitive verbs \cite{chtrans}. For each of these proposed definitions I will provide counterexamples to show that, while they may accurately describe some classes of words in a particular language, they are not universal definitions.

\section{Outline}

\begin{itemize}
\item Introduction
\item Orthographic Words, reviewing the first definition proposed (separating words by spaces).
\item Phonological Words, reviewing separating words by pauses.
\item The Expression of an Idea, reviewing the definition Harley takes from the Oxford English Dictionary.
    \begin{itemize}
    \item Discussion phonemes and allophones
    \item Examining how it explains clitics
    \item Discussing meaning and transitivity of verbs
    \end{itemize}
\end{itemize}

\section{Bibliography}

\printbibliography
\end{document}
