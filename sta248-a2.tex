\documentclass{article}
\usepackage{amsmath,amssymb,graphicx,tikz}
\usepackage{hyperref}
\usepackage{tikz}
\usepackage{amsfonts}
\usepackage{amsmath}
\usepackage{listings}
\usepackage{pgfplots}
\usepgfplotslibrary{statistics}

\title{STA248 Winter 2016, Assignment 1}
\author{Connor Peet \#1001088208}
\renewcommand{\today}{~}
\hypersetup{pdfpagemode=Fullscreen,
  colorlinks=true,
  linkfileprefix={}}
\begin{document}
\maketitle

\lstset{
    numbers=left
}

\newcommand{\overbar}[1]{\mkern 1.5mu\overline{\mkern-1.5mu#1\mkern-1.5mu}\mkern 1.5mu}

\begin{enumerate}
\item [1.]
    \begin{enumerate}
    \item [(a)] $\overbar{X} \sim \mathcal{N}(20, \sqrt{\frac{25}{9}}) = \overbar{X} \sim \mathcal{N}(20, \frac{5}{3})$
    \item [(b)] \vspace{3cm}
    \item [(c)] $\alpha = P(\overbar{X} > 25 | \mu = 20) = P(Z > \frac{25 - 20}{5 / 3}) \approx 0.001350$
    \item [(d)] $\overbar{X} \sim \mathcal{N}(28, \sqrt{\frac{25}{9}}) = \overbar{X} \sim \mathcal{N}(28, \frac{5}{3})$
    \item [(e)] \vspace{3cm}
    \item [(f)] $\beta = P(\overbar{X} < 25 | \mu = 28) = P(Z < \frac{25 - 28}{5 / 3}) \approx 0.03593$
    \item [(g)] $\text{Power} = 1 - \beta = 0.9641$
    \item [(h)] When the same size is increased, the standard deviation will increase in a squared relationship. This will cause the two curves to become `skinnier' and more centralized close to their mean.
    \item [(i)] If the size is increased but the critical point is not changed, $\alpha$ and $\beta$ will decrease.
    \end{enumerate}
\item [2.]
    \begin{enumerate}
    \item [(a)] $\hat{p} = \frac{75}{193} \approx 0.3887 $
    \item [(b)]
        \begin{equation*}
        \begin{aligned}
        \text{Interval} =& (\hat{p} - z_{\alpha / 2} \sqrt{\hat{p} (1 - \hat{p}) / n}, \hat{p} + z_{\alpha / 2} \sqrt{\hat{p} (1 - \hat{p}) / n} \\
            =& (0.3887 - 1.96 \sqrt{0.3887 (1 - 0.3887) / 193}, 0.3887 + \\
            & \qquad 1.96 \sqrt{0.3887 (1 - 0.3887) / 193} \\
            =& (0.3199, 0.4575)
        \end{aligned}
        \end{equation*}
    \item [(c)]
        \begin{equation*}
        \begin{aligned}
        d =& z_{\alpha / 2} \sqrt{\hat{p} (1 - \hat{p}) / n} \\
        0.3 =& 2.575 \sqrt{0.3887 (1 - 0.3887) / n} \\
        n =& 17.51
        \end{aligned}
        \end{equation*}
    \end{enumerate}
\item [3.]
    \begin{enumerate}
        \item
            \begin{equation*}
            \begin{aligned}
                & H_0: \sigma_1^2 = \sigma_2^2, H_1: \sigma_1^2 \neq \sigma_2^2,  \\
                & S_1^2 / S_2^2 \approx 1.0969 \\
                & P(F_{60, 60} > 1.395) = 0.10 \\
                & \text{Then we cannot reject }H_0\text{, their variances may be equal}.
            \end{aligned}
            \end{equation*}
        \item $S^2_p = \frac{(n_1 - 1)S^2_1 + (n_2 - 1)S^2_2}{n_1 + n_2 - 2} = \frac{(61 - 1)24.9 + (61 - 1)22.7}{61 + 61 - 2} = 23.8$
        \item
            \begin{equation*}
            \begin{aligned}
            \Delta =& t_{\alpha / 2} \sqrt{S^2_p(1 / n_1 + 1 / n_2)} \\
                =& 1.658 \sqrt{23.8 (1 / 61 + 1 / 61)} = 1.465\\
            \text{Interval} =& ((\overbar{x_1} - \overbar{x_2}) - \Delta, (\overbar{x_1} - \overbar{x_2}) + \Delta) \\
                =& (11 - 1.465, 11 + 1.465) \\
                =& (9.535, 12.465)
            \end{aligned}
            \end{equation*}
        \item The new laser definitely reads more barcodes than the old one. We're 95\% certain that it reads more than nine barcodes per second more.
        \item The central limit theorem.
    \end{enumerate}
\item [4.] \begin{equation*}
        \begin{aligned}
            S^2_p =& \frac{(n_1 - 1) S_1^2 + (n_2 - 1) S_2^2}{n_1 + n_2 - 2} \\
                & \text{But, we know the underlying deviation }\sigma\text{ is equal, so:} \\
            S^2_p =& \frac{(n_1 - 1) \sigma^2 + (n_2 - 1) \sigma^2}{n_1 + n_2 - 2} \\
            S^2_p =& \frac{\sigma^2(n_1 - 1 + n_2 - 1)}{n_1 + n_2 - 2} \\
            S^2_p =& \frac{\sigma^2(n_1 + n_2 - 2)}{n_1 + n_2 - 2} \\
            S^2_p =& \sigma^2 \\
        \end{aligned}
        \end{equation*}
\item [5.]
    \begin{enumerate}
    \item Differences: \\
        \begin{lstlisting}
[1] -0.03 -0.08  0.02 -0.11 -0.03 -0.04
[7] -0.07 -0.01 -0.14 -0.08 -0.04 -0.03
[13] -0.07 -0.04 -0.07
        \end{lstlisting}
    \item $\overbar{d} = -0.05467$, $s_d = 0.03998$.
    \item
        \begin{equation*}
        \begin{aligned}
        \text{Interval} =& (\overbar{d} - t_{\alpha / 2} s_d / \sqrt n, \overbar{d} + t_{\alpha / 2} s_d / \sqrt n) \\
            =& (-0.05467 - 1.761 \times 0.03998 / \sqrt{15}, -0.05467 + 1.761 \times 0.03998 / \sqrt{15}) \\
            =& (-0.07285, -0.03649)
        \end{aligned}
        \end{equation*}
    \end{enumerate}
\end{enumerate}


\newpage
\appendix
\section{Code for Question 5(a) and 5(b)}
\lstinputlisting[language=R, breaklines=true]{sta248-a2-q5.r}

\end{document}
