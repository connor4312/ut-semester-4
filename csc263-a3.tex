\documentclass{article}
\usepackage{amsmath,amssymb,graphicx,tikz}
\usepackage{hyperref}
\usepackage{tikz}
\usepackage{amsfonts}
\usepackage{amsmath}
\usepackage{listings}

\title{CSC263 Winter 2016, Assignment 2}
\author{Connor Peet \#1001088208}
\renewcommand{\today}{~}
\hypersetup{pdfpagemode=Fullscreen,
  colorlinks=true,
  linkfileprefix={}}
\begin{document}
\maketitle

\lstset{
    numbers=left
}

\begin{enumerate}
\item [1.]
    \begin{enumerate}
    \item Each node of the AVL tree will contain an integer corresponding to an item in the set $S$. As augmented axillary information it will also contain the sums of all items in its left and right subtrees, or zero if it has no corresponding subtree. A few modifications to \texttt{AVL-Insert(S, i)} to maintain this, none of which alter the worst-case run time of the function:
        \begin{itemize}
        \item If, when running \texttt{AVL-Insert}, we encounter a node that already contains $i$, abort the insertion.
        \item After \texttt{AVL-Insert} is called and before rebalancing, recurse from the new leaf up the tree and update parent nodes' sums. This will do $O(\log n)$ operations (the height of the tree) so will not change the asymptotic complexity.
        \item Rotation functions can be augmented to update nodes' sums in constant time.
        \end{itemize}
    \item Addition has already been described above: it's simply an insertion into the AVL tree. To calculate the sum, one does a standard search on the binary tree, but with a counter initialized to zero. When you traverse down a node add the node's value to the counter, and when you transverse to a right subtree, add the parent's left subtree sum to the counter. When you reach the node, return the value stored in the counter. Because this is just a regular binary search with some constant-time record-keeping at each recursion, it can be expected to run in $O(\log n)$.

\begin{lstlisting}
def sum(T, i, counter=0):
    if T.root == i:
        return T.leftSum + counter
    elif i < T.root:
        if T.left is not None:
            return sum(T.left, i, counter + T.root)
        else:
            return counter
    elif i > T.root:
        if T.right is not None:
            return sum(T.right, i, counter + T.root + T.leftSum)
        else:
            return counter + T.root + T.leftSum
\end{lstlisting}
    \end{enumerate}

\item [2.]
    \begin{enumerate}
    \item [a.] To solve the problem, we'll keep a hash map acting as a set of integers. We insert all $x \in X$ into a hashmap $H$. Then for every $y \in Y$, we'll check if there's any value $z - y_i$ already in the set. If there does exist such an item in the set, we know that $\exists x \in X, z = x + y_i$.

\begin{lstlisting}
def check(X, Y, z):
    H = new Hashmap()
    for x in X:
        H.insert(x, True)

    for y in Y:
        if H.get(z - y) == True:
            return True

    return False
\end{lstlisting}


    \item [b.] The algorithm is fairly simple, running at most $2n$ calls to \texttt{Hashmap.insert} and \texttt{Hashmap.get} in the case that there does not exist an $x$ and $y$ to satisfy the requirements. From the text, the expected running time of insert and retrieval on hashmaps is $O(1)$, therefore the expected running time of this algorithm is in $O(n)$.

    \item [c.] In the worst case, the stars align and every integer in $X$ and $Y$ hash to the same key, and we end up transversing the linked list at that key. The worst-case running time of insert is still $O(1)$, but the retrieval is $O(n)$. Insertion is run $n$ times, so the worst-case running time of this algorithm is $O(n^2)$.

    \end{enumerate}
\end{enumerate}
\end{document}
